\documentclass[12pt]{article}
\usepackage{hyperref}
\usepackage{graphicx}
\usepackage[table,xcdraw]{xcolor}
\usepackage{listings}
\usepackage{float}
\usepackage{fixltx2e}
\restylefloat{table}
\graphicspath{ {images/} }

\sloppy
\definecolor{lightgray}{gray}{0.5}
\setlength{\parindent}{0pt}

\definecolor{darkgreen}{rgb}{0,0.6,0}
\definecolor{ident}{rgb}{.3,.0,.2}


\lstset{language=matlab}
\lstset{showstringspaces=false}
\lstset{frame=L}
\lstset{breaklines=true}
\lstset{basicstyle=\footnotesize\ttfamily\bfseries}
\lstset{commentstyle=\color{darkgreen}}
\lstset{keywordstyle=\bfseries\color{blue}}
\lstset{numberstyle=\color{black}}
\lstset{rulecolor=\color{black}}
\lstset{numbers=left}
\renewcommand{\thefootnote}{\fnsymbol{footnote}}

\title{Quadcopter Design and Build Log}
\author{M Dobosz}

\begin{document}
\begin{titlepage}
\vspace*{-4.5cm}
\hspace*{-4.1cm}
{\let\newpage\relax\maketitle}

\begin{table}[b]
\begin{center}
\begin{tabular}{| l | l | l |}
\hline
\textbf{Revision Date} & \textbf{Description} \\ \hline
01/09/18 & Initial document \\ \hline
02/11/18 & Update to use Mavic frame \\ \hline
& \\ \hline

\end{tabular}
\end{center}
\end{table}
\end{titlepage}


\pagebreak
\tableofcontents
\pagebreak

\section{Introduction}

The purpose of this document is to record and explain the design process of the quadcopter and why each decision was made. Since I have little experience with the design process, all steps performed for this project were made up as I went along and may or may not be representative of the real-world design process. This document was created to better organize my thoughts and each decision I made for future collaborators to get up to speed and to learn what is going on. All information that is obtained from outside sources is referenced with footnotes, however only the url or website is noted and no proper citation format is used. All information is presented in a way that someone with little to no technical experience can understand (hopefully).
\\

A quadcopter is a four rotor manned or unmanned vehicle used mostly for aerial photography. Many sensors such as gyroscopes and accelerometers, which are located on the flight controller, are used to provide the aircraft with feedback on altitude, acceleration, orientation, etc. The "heart" of the quadcopter is the flight controller, which takes data from its sensors and the receiver and sends the information to the ESCs\footnote{Electonic Speed Controller; used to convert the DC voltage from the flight controller to a three-phase AC singal for the motors} More info can be found \href{https://en.wikipedia.org/wiki/Quadcopter}{\color{cyan}here}.
\\

All prices are presented in \$CAD.

\section{Objectives}

The main objectives of the project are:
\renewcommand{\labelitemi}{\textperiodcentered}
\begin{itemize}
\item a minimum of 10 minutes of flight time
\item easily replaceable parts in the event of a failure or crash
\item small enough to carry in a backpack
\end{itemize}

All decisions that must be made for which materials to use, physical design, etc. will follow these rules strictly.
\\

Secondary objectives include:
\renewcommand{\labelitemi}{\textperiodcentered}
\begin{itemize}
\item GPS support
\item sonar for low-altitude hold
\item space and bracket(s) for mounting of a GoPro camera in the future for aerial photography
\end{itemize}

Two main design sections will be considered, the frame and the electronics. The frame will consist of the structural components of the quadcopter where all the electronic components will be mounted. The electronic components will include the flight controller, motor, battery and receiver. 
\\

The main restriction imposed on this project is cost. Most, if not all, decisions are made with cost as one of the heaviest factors. Time and difficulty are not as important as this project does not have a deadline and new skills can be learned along the way.
\\

For all custom components, either SolidWorks or Altium Designer will be used.

\section{Frame Design}

The frame design is the section that will cause the most trouble, as I have little to no mechanical experience and designing a frame that is both lightweight and durable is an entire project in itself. To simplify the designing of the frame, CdRsKuLL's Mavic F3 v4.1 frame from Thingiverse will be used.

\subsection{Frame Material}

According to Simplify3D, the estimated volume of the frame when 3D printed will be 280 cm\textsuperscript{3}. Depending on the material chosen, a weight can be calculated and suitable motors can be selected.
\\

Five materials were considered for the frame design: PLA (Polylactic Acid), ABS (Acrylonitrile Butadiene Styrene), Carbon Fiber PLA and Nylon and PETG (Polyethylene Terephthalate Glycol). The advantages and disadvantages of each material were investigated and are presented below.
\\

\textbf{PLA:}
\renewcommand{\labelitemi}{\textperiodcentered}
\begin{itemize}
\item[+] easy to print with
\item[+] biodegradable
\item[-] warps if exposed to sunlight for too long
\item[-] lower structural stability compared to other materials
\item[Price:] \$31 per kg 
\item[Density:] 1.25g/cm\textsuperscript{3}
\end{itemize}


\textbf{ABS:}
\renewcommand{\labelitemi}{\textperiodcentered}
\begin{itemize}
\item[+] high durability
\item[+] flexible and lightweight
\item[-] unpleasant fumes released when printing
\item[-] prone to warping without heated enclosure
\item[Price:] \$30 per kg
\item[Density:] 1.05g/cm\textsuperscript{3}
\end{itemize}

\textbf{CF PLA:}
\renewcommand{\labelitemi}{\textperiodcentered}
\begin{itemize}
\item[+] very high durability mimicking carbon fiber
\item[+] prints like PLA
\item[-] required hardened nozzle
\item[-] expensive
\item[Price:] \$60 per kg + nozzle
\item[Density:] 1.30g/cm\textsuperscript{3}
\end{itemize}

\textbf{Nylon:}
\renewcommand{\labelitemi}{\textperiodcentered}
\begin{itemize}
\item[+] strong. durable and flexible
\item[+] less brittle than PLA and ABS
\item[-] high temperature required for printing
\item[-] emits toxic fumes when printing
\item[Price:] \$50 per kg
\item[Density:] 1.15g/cm\textsuperscript{3}
\end{itemize}

\textbf{PETG:}
\renewcommand{\labelitemi}{\textperiodcentered}
\begin{itemize}
\item[+] high durability and flexibility
\item[+] does not shrink
\item[+] does not warp
\item[-] requires fine tuning of printer settings
\item[Price:] \$40 per kg
\item[Density:] 1.27g/cm\textsuperscript{3}
\end{itemize}


\section{Electronics}

\end{document}
